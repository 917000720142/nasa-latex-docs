%%%%%%%%%%%%%%%%%%%%%%%%%%%%%%%%%%%%%%%%%%%%%%%%%%%%%%%%%%%%%%%%%%%%%%%%%%%%%%%%%%%%%%%%%%%%%%%%%%%
% TEMPLATE: Define bibliography options
%%%%%%%%%%%%%%%%%%%%%%%%%%%%%%%%%%%%%%%%%%%%%%%%%%%%%%%%%%%%%%%%%%%%%%%%%%%%%%%%%%%%%%%%%%%%%%%%%%%

% Define the options for bibliography and load biblatex package
\RequirePackage[
   backend     = bibtex8, 
   giveninits  = true, 
   natbib      = true, 
   style       = aiaa, 
   autocite    = superscript, 
   bibwarn     = false, 
   biblabel    = superscript]{biblatex}

% Allows cite command to control autocite
\let\cite\autocite

%%%%%%%%%%%%%%%%%%%%%%%%%%%%%%%%%%%%%%%%%%%%%%%%%%%%%%%%%%%%%%%%%%%%%%%%%%%%%%%%%%%%%%%%%%%%%%%%%%%
% TEMPLATE: Define document format settings
%%%%%%%%%%%%%%%%%%%%%%%%%%%%%%%%%%%%%%%%%%%%%%%%%%%%%%%%%%%%%%%%%%%%%%%%%%%%%%%%%%%%%%%%%%%%%%%%%%%

% Set the geometry for each page
\geometry{margin=1in,includefoot} 

% Load times package to change document font to Times New Roman
\RequirePackage{times}

% Load mathptmx to change the equation font to Times New Roman
\RequirePackage{mathptmx}

% Change the fontsizes for a 10pt base (skip 1.2 x fontsize) (default in nasa-latex-docs.cls = 11pt)
% Link: https://en.wikibooks.org/wiki/LaTeX/Fonts#Font_families
\renewcommand\tiny{\@setfontsize\tiny{5pt}{6pt}}
\renewcommand\scriptsize{\@setfontsize\scriptsize{7pt}{8.4pt}}
\renewcommand\footnotesize{\@setfontsize\footnotesize{8pt}{9.6pt}}
\renewcommand\small{\@setfontsize\small{9pt}{10.8pt}}
\renewcommand\normalsize{\@setfontsize\normalsize{10pt}{12pt}}
\renewcommand\large{\@setfontsize\large{12pt}{12pt}}
\renewcommand\Large{\@setfontsize\Large{14.4pt}{17.28pt}}
\renewcommand\LARGE{\@setfontsize\LARGE{17.28pt}{20.74}}
\renewcommand\huge{\@setfontsize\huge{20.74pt}{24.88}}
\renewcommand\Huge{\@setfontsize\Huge{24.88pt}{29.86}}

% Set the depth of the sections
\setcounter{secnumdepth}{3}

% Define section font styling
\titleformat{\section}{\centering\normalfont\bfseries\fontsize{11}{12}\selectfont}{\thesection}{1em}{}
\titleformat{\subsection}{\normalfont\bfseries\normalsize}{\thesubsection}{1em}{}
\titleformat{\subsubsection}{\normalfont\itshape\normalsize}{\thesubsubsection}{1em}{}

% Renew the commands for proper section numbering
\renewcommand{\thesection}{\Roman{section}.} 
\renewcommand{\thesubsection}{\Alph{subsection}.} 
\renewcommand{\thesubsubsection}{\arabic{subsubsection}.} 

% Sets the footnote number styles to symbols - create new symbol to wrap around first 9
\newalphalph{\fnsymbolwrap}[wrap]{\@fnsymbol}{}
\renewcommand{\thefootnote}{\fnsymbolwrap{\value{footnote}}}

% Define the distance between footer (page number) and text
\setlength{\footskip}{2\baselineskip}

% Define the headheight - set to zero, no header in this format
\setlength{\headheight}{0pt}
\setlength{\headsep}{0pt}
\setlength{\topmargin}{0pt}

% Modify the header and footer lines - set to zero to remove
\renewcommand{\headrulewidth}{0pt}
\renewcommand{\footrulewidth}{0pt}

% Change all quoting environments to be 9pt font
\AtBeginEnvironment{quote}{\normalsize}
\AtBeginEnvironment{quotation}{\small}
\AtBeginEnvironment{verse}{\small}

% Change the figure and table label separation characters to periods
\captionsetup[table]{labelsep=period}
\captionsetup[figure]{labelsep=period}

%%%%%%%%%%%%%%%%%%%%%%%%%%%%%%%%%%%%%%%%%%%%%%%%%%%%%%%%%%%%%%%%%%%%%%%%%%%%%%%%%%%%%%%%%%%%%%%%%%%
% TEMPLATE: Define document header and footer settings
%%%%%%%%%%%%%%%%%%%%%%%%%%%%%%%%%%%%%%%%%%%%%%%%%%%%%%%%%%%%%%%%%%%%%%%%%%%%%%%%%%%%%%%%%%%%%%%%%%%

% Create the Footer - Clear the current header/footer style
\fancyhf{} 

% Set the current header/footer style
\pagestyle{fancy}

% Define the centered footer
\cfoot{\thepage \\ American Institute of Aeronautics and Astronautics}

%%%%%%%%%%%%%%%%%%%%%%%%%%%%%%%%%%%%%%%%%%%%%%%%%%%%%%%%%%%%%%%%%%%%%%%%%%%%%%%%%%%%%%%%%%%%%%%%%%%
% TEMPLATE: Define custom content commands to be used in template file(s)
%%%%%%%%%%%%%%%%%%%%%%%%%%%%%%%%%%%%%%%%%%%%%%%%%%%%%%%%%%%%%%%%%%%%%%%%%%%%%%%%%%%%%%%%%%%%%%%%%%%

\newcommand{\aiaaTitle}{
   \begin{center}
      % Create title: 18pt font, bold, centered 
      \LARGE\bfseries 
      \@docTitle \vspace{\baselineskip}
   \end{center}
}

\newcommand{\aiaaAuthors}{
   \begin{center}
      \multido{\ii=1+1}{10}{%
         \checkdocAuthorName(\ii)
         \ifemptydata
            % \docAuthor not defined for this index, skip
         \else
            % Funky workaround to get "\thanks" to ready array - sets value to "\cachedata"\setlength{\leftskip}{0.5\baselineskip}
            \checkdocAuthorPosition(\ii) 
            \docAuthorName(\ii)\thanks{\small\cachedata}\break
            % \docAuthorName(\ii)\break
            \textit{\docAuthorOrganization(\ii), \docAuthorLocation(\ii)} \break
            % Determine if there is an author after the current index to print "and"
            \edef\iinext{\number\numexpr\ii+1\relax}
            \checkdocAuthorName(\iinext)
            \ifemptydata
               \break
            \else
               \\ \vspace{-0.5\baselineskip} \textit{and} \vspace{0.5\baselineskip} \break
            \fi    
         \fi   
      }
   \end{center}
}

\newcommand{\aiaaAbstract}{
   % Create abstract by putting minipage in centered box
   \noindent\makebox[\textwidth][c]{
   % Letter paper width - left margin - right margin - 0.5 inch indent on each side (left/right)
   % 8.5in - 1in - 1in - 0.5in - 0.5in = 5.5 inch width for abstract
   \begin{minipage}{5.5in} 
      \normalsize\bfseries
      \@docAbstract 
   \end{minipage}}
}