% Define a custom color
\definecolor{backcolour}{rgb}{0.95,0.95,0.92}
\definecolor{codegreen}{rgb}{0,0.6,0}

% Define a custom style
\lstdefinestyle{myStyle}{
    backgroundcolor=\color{backcolour},   
    commentstyle=\color{codegreen},
    basicstyle=\ttfamily\footnotesize,
    breakatwhitespace=false,         
    breaklines=true,                 
    keepspaces=true,                 
    numbers=left,       
    numbersep=5pt,                  
    showspaces=false,                
    showstringspaces=false,
    showtabs=false,                  
    tabsize=2,
}

% Use \lstset to make myStyle the global default
\lstset{style=myStyle}

\lstinputlisting[caption=Sample Code Listing C++, label={lst:listing-cpp}, language=C++]{code_sample.cpp}

\lstinputlisting[caption=Sample Code Listing Python, label={lst:listing-python}, language=Python]{code_sample.py}

\lstinputlisting[caption=Sample Code Listing Matlab, label={lst:listing-matlab}, language=Matlab]{code_sample.m}

We can reference a range of tables as seen here: \refrange{lst:listing-cpp}{lst:listing-matlab}.
Also note how the "Listings" prefix is automatically added within the document text whenever the range reference is called.